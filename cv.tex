\documentclass[11pt,a4paper]{moderncv}
\moderncvtheme[blue,sans]{casual}

\usepackage[utf8]{inputenc}
\usepackage{fontawesome}
\usepackage[scale=0.8]{geometry}
\renewcommand{\cvcomputer}[2]{\cvline{#1}{\small#2}}
\newcommand\Colorhref[3][magenta]{\href{#2}{\small\color{#1}#3}}

\usepackage{xltxtra}
\usepackage{xgreek}

\usepackage[english,greek]{babel}

\setsansfont{Arial}
\setmonofont{Courier New}
\setmainfont[Mapping=tex-text]{Times New Roman}
\setlength{\parskip}{1em}


%% Character encoding

%% Adjust the page margins
\usepackage[scale=0.8]{geometry}

%% Personal data
\firstname{\Huge Απόστολος}
\familyname{\Huge Καλαμπούκας}
\title{\textit{Βιογραφικό Σημείωμα}, Junior Software Developer}
\address{Διεύθυνση κατοικίας}{Βικάτου 5, Αμπελόκηποι,115 24 - Αθήνα, Ελλάδα}
\mobile{+(0030) 697 2695 779}
%\phone{555-vulcan}
\email{akalamboukas@yahoo.com}
%\homepage{startrek.com/database\_article/spock}
%\lattes{lattes.cnpq.br/2222077812442414}
%\extrainfo{additional information (optional)}
%\photo[64pt]{picture}                         % '64pt' is the height
%\quote{Some quote (optional)}
% to show numerical labels in the bibliography; only useful if you make
% citations in your resume
%\homepage{www.johndoe.com}
\social[linkedin][gr.linkedin.com/in/apostolos-kalampoukas-36779040]{LinkedIn}
\social[github][github.com/ApostolisKala]{GitHub}
\social[twitter][twitter.com/ApostolisKala]{Twitter}
\social[skype][skype.com]{Skype}

\extrainfo{Ημ. Γέννησης : 18 Ιαν 1990, 
Στρατιωτικές Υποχρεώσεις: Εκπληρωμένες }
%\photo[64pt][0.4pt]{picture}
%\quote{Some quote (optional)}

\begin{document}
\makecvtitle
\section{Εκπαίδευση}
\cventry{Ιούνιος 2008}{Απόφοιτος 4ου Γενικού Λυκείου Καρδίτσας}{}{}{}{}
\cventry{2008-2016}{Απόφοιτος του Τμήματος Επιστήμης και Τεχνολογίας Υπολογιστών}{Πανεπιστήμιο Πελοποννήσου}{Κατεύθυνση Συστημάτων Λογισμικού}{GPA 6.93 }{Τρίπολη, Ελλάδα}

\section{Εργασιακή εμπειρία}
\cventry{Ιανουάριος 2016 - μέχρι σήμερα}{Freelance}{}{}{}{Παραδίδω μαθήματα σε προπτυχιακούς φοιτητές στα μαθήματα Δομές Δεδομένων, C, Java, Γλώσσες Περιγραφής Υλικού και Λειτουργικά Συστήματα. Επίσης, αναλαμβάνω την κατασκευή ιστοσελίδων σε πλατφόρμες Wordpress, Joomla καθώς και custom ιστοσελίδες σε html5.}
\cventry{Οκτώβριος 2013 - Αύγουστος 2014}{GlaphX Ltd}{StartUp Εταιρεία Πληροφορικής στην Αθήνα, Ελλάδα}{}{}{Απασχολήθηκα στην εταιρεία ΓΛΑΥΞ – ΜΟΝΟΠΡΟΣΩΠΗ ΙΚΕ ως Προγραμματιστής/ Web Developer. Τα καθήκοντα μου ήταν η συγγραφή κώδικα σε γλώσσες προγραμματισμού Java, C, PHP, CSS, HTML5, AJAX, Javascript καθώς επίσης και η δημιουργία responsive ιστοσελίδων σε έτοιμες πλατφόρμες του Joomla, του Wordpress  και Drupal. Επίσης, χρησιμοποιώντας τις παραπάνω γλώσσες προγραμματισμού και με τη βοήθεια του \emph{Codeigniter Framework} ανέλαβα την υλοποίηση τμημάτων του HRM ( Human Resource Management), συστήματος που χρησιμοποιεί η εταιρεία. (Διαθέσιμη Συστατική Επιστολή από την εταιρεία)}
\cventry{2007 - 2010  Περίοδοι Καλοκαιριού}{Υπεύθυνος Σχολής Οδηγών}{}{}{}{Απασχολήθηκα στην Σχολή Οδηγών προσφέροντας διοικητικές υπηρεσίες και ανέλαβα πλήρως την λειτουργία και εκμάθηση των υπολογιστών στους υποψήφιους μαθητές για την προετοιμασία τους στο θεωρητικό κομμάτι των εξετάσεων των σημάτων οδήγησης. Οι υπολογιστές είχαν ένα συγκεκριμένο εκπαιδευτικό πρόγραμμα πάνω στο οποίο ήταν πιο εύκολη η εκμάθηση των σημάτων και του κώδικα οδικής κυκλοφορίας με φωτογραφίες που απεικόνιζαν διάφορα παραδείγματα}
\cventry{Ιούνιος 2006 - Αύγουστος 2006}{Μαθητευόμενος Hardware}{}{}{}{Τη περίοδο του καλοκαιριού 2006 είχα απασχοληθεί για 2 περίπου μήνες σε ένα μαγαζί υπολογιστών όπου κύρια δραστηριότητά μου ήταν η αντικατάσταση χαλασμένων αντικειμένων των υπολογιστών (Laptop, Desktop). Κυρίως, επεξεργαζόμουν τους υπολογιστές και έκανα την αντικατάσταση χαλασμένων καρτών γραφικών, επεξεργαστών, σκληρών δίσκων και άλλα , με καινούργιους}


\section{Γλώσσες}
\cvlanguage{Ελληνικά}{Μητρική Γλώσσα}{}
\cvlanguage{Αγγλικά}{Πολύ Καλά}{ECCE, University of Michigan}{}

\section{Κοινωνικές δεξιότητες και ικανότητες}

\subsection{Ομαδικότητα}
\cvline{}{Έχω εργαστεί σε διαφόρων τύπων ομάδες από τις ερευνητικές ομάδες στο   Πανεπιστήμιο Πελοποννήσου.}{}

\subsection{Μεταδοτικότητα}
\cvline{}{Στα πλαίσια του μαθήματος Γλώσσες Περιγραφής Υλικού του Πανεπιστημίου Πελοποννήσου μου είχε ανατεθεί πλήρως το εργαστήριο Ψηφιακής Σχεδίασης από τον Καθηγητή Κ. Καββαδία Νικόλαο. Σκοπός μου ήταν η εκπαίδευση των πρωτοετών και δευτεροετών συμφοιτητών μου πάνω στην γλώσσα Verilog και την εκπόνηση πειραμάτων και εξαγωγής συμπερασμάτων από κάθε εργασία που τους είχε αναθέσει ο ίδιος ο καθηγητής εβδομαδιαία.(Διαθέσιμη και βεβαίωση από τον Καθηγητή)}{}

\section{Οργανωτικές δεξιότητες και ικανότητες}
\cvline{}{Κατά τη διάρκεια της περιόδου που βρισκόμουν στην Τρίπολη είχα ασχοληθεί με την ομάδα εργασίας της ACM του Πανεπιστημίου Πελοποννήσου για θέματα αλληλεπίδρασης ανθρώπου υπολογιστή που δημιουργήθηκε το Σεπτέμβριο 2008 για να προωθήσει τη συνεργασία μεταξύ Ελληνικών ερευνητικών και τεχνολογικών φορέων, επιχειρήσεων, δημόσιων φορέων που έχουν κοινά ενδιαφέροντα στο πεδίο της αλληλεπίδρασης ανθρώπου-υπολογιστή.}{}


  
\section{Σεμινάρια-Ημερίδες-Συνέδρια}
\cvline{}{\begin{itemize}
			\item Σεμινάριο Μονάδας Καινοτομίας και επιχειρηματικότητας Πανεπιστημίου Πελοποννήσου(14/12/2012).
			\item Ημερίδα με θέμα “Μεταπτυχιακές σπουδές στο εξωτερικό – Υποτροφίες για μεταπτυχιακές σπουδές στο εξωτερικό”(4/12/1012).
			\item Συνέδριο «1st Coach a Winning Team Conference» (26/04/2013).
			\item Job Fair Athens 2013 - Διημερίδα Καριέρας (3\&4 /04/2013).
			\item Πανευρωπαϊκό Δίκτυο EURAXESS Researchers in Motion (15/05/2013).
			\item Ανοιχτό σεμινάριο Καινοτομίας (06/06/2013).
			\item Social Media \& Internet Marketing at InnovAthens.
			\item Colab innovation (01/02/2014).
			\item Angular JS (07/02/2014) at Colab Athens.
	    \end{itemize}}{}
	    
\section{Τεχνικές δεξιότητες και ικανότητες}
\cvcomputer{Λειτουργικά Συστήματα}{Unix/Linux, Windows}
\cvcomputer{Γλώσσες Προγραμματισμού}{\textsf{Java}, PHP, Objective PHP, HTML, CSS, C, C++, \LaTeX{}, UML, XML, Javascript, Android Programming, Web Services (REST)  }
\cvcomputer{Λογισμικό Παραγωγής}{Microsoft Office, Adobe Photoshop, Adobe Dreamweaver, Adobe Flash, Adobe Illustrator, GIMP, Netbeans, Android Studio, Eclipse}
\cvcomputer{Πλατφόρμες}{WordPress, Joomla, Drupal}
\cvcomputer{MVC Frameworks}{Codeigniter}
\cvcomputer{Διάφορα}{git, Shell, Partial knowledge in software security systems}


\section{Projects και εργασίες}
\cvline{}{\begin{itemize}
		\item\Colorhref{http://gpetritsis.gr/}{http://gpetritsis.gr/}
 	           \item\Colorhref{http://kalamboukas.gr/}{http://kalamboukas.gr/}
  		\item\Colorhref{http://adampetritsis.com/}{http://adampetritsis.com/}
  		\item\Colorhref{http://s-psifis.gr/}{http://s-psifis.gr/}
  		\item\Colorhref{http://dm-engineering.gr/}{http://dm-engineering.gr/}
  		\item Το θέμα της εργασίας είναι ο σχεδιασμός, η ανάλυση και η υλοποίηση μιας βάσης δεδομένων, όπου θα καταχωρούνται τα στοιχεία από τις καλύτερες κινηματογραφικές ταινίες όλων των εποχών (σύμφωνα με το IMDB), τα είδη στα οποία ανήκουν, τα στοιχεία των σκηνοθετών τους και τα στούντιο παραγωγής τους. (Μάθημα : Βάσεις Δεδομένων)
  		\item Στην εργασία αυτή θα πρέπει να εκτελέσετε συγκεκριμένες από τις φάσεις κύκλου ζωής ενός
           συστήματος καταγραφής και διαχείρισης εργασιών. Οι φάσεις που θα πρέπει να εκτελεστούν είναι
            οι ακόλουθες:
            \begin{itemize}
		\item ανάλυση απαιτήσεων
		\item σχεδιασμός συστήματος
		\item (μερική) υλοποίηση συστήματος   (Μάθημα : Τεχνολογία Λογισμικού)
 	  \end{itemize}
 	  	\item Στην άσκηση αυτή καλείστε να υλοποιήσετε το Διανυσματικό Μοντέλο Ανάκτησης Πληροφορίας σε κοινωνικά δίκτυα. Η γλώσσα υλοποίησης μπορεί να είναι η C, C ++ ή Java. (Μάθημα : Ανάκτηση Πληροφορίας)
 	  	\item Στην εργασία αυτή καλείστε να υλοποιήσετε προγράμματα για μια δικτυακή εκδοχή του γνωστού
             επιτραπέζιου παιχνιδιού «Γκρινιάρης».(Μάθημα : Προηγμένα Θέματα Προγραμματισμού)
             	\item Σκοπός αυτής της εργασίας είναι να κατανοήσετε κάποιες βασικές έννοιες και τεχνικές φτιάχνοντας ένα δικό σας Σύστημα Ανάκτησης Πληροφορίας το οποίο θα χρησιμοποιεί το Boolean μοντέλο και το οποίο θα ονομάσετε όπως σας αρέσει.(Μάθημα : Ανάκτηση Πληροφορίας)
             	\item \textbf{Πτυχιακή Εργασία}: Υλοποίηση ολοκληρωμένου συστήματος διαχείρισης πρακτικών συνελεύσεων μέσω διαδικτύου και κινητών συσκευών . Η εργασία αυτή δημιουργήθηκε για να επιλύσει καθημερινά προβλήματα εντός της Πανεπιστημιακής Κοινότητας. Για την υλοποίησή της χρησιμοποιήθηκε το MVC Framework Codeigniter για το Web κομμάτι καθώς και μία εφαρμογή πάνω στην πλατφόρμα του Android. Η πλήρης αναφορά της πτυχιακής εργασίας δημιουργήθηκε σε \LaTeX{}.
	    \end{itemize}}{}
	    
\section{Ενδιαφέροντα}	    

\cvline{}{Τον ελεύθερο χρόνο μου φροντίζω να τον αξιοποιώ με δραστηριότητες που συμβάλλουν στην αποφόρτιση που συσσωρεύει η καθημερινότητα και αυτές είναι το διάβασμα, το ποδόσφαιρο, το γυμναστήριο και το χειμώνα το snowboard.}{}
	    
\end{document}